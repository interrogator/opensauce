%!TEX root = ../sauce_book.tex

% recipe title
\recipe{Spiced yoghurt sauce}

% author in small caps
Submitted by: \textsc{Anna Varghese, University of Melbourne}

% enticement:

\begin{shadequote*}
A traditional recipe from Kerala.
\end{shadequote*}
\hrulefill

% ingredients:
\textbf{Ingredients}

\begin{enumerate}[before=\itshape,font=\normalfont]
\item 1--2 cups Greek yoghurt
\item 1 tsp mustard seeds
\item 1 tsp cumin
\item 1 tsp turmeric
\item \textonehalf
\item Garlic (crushed)
\item Grated ginger (one small knob)
\item Vegetable oil
\item 10 curry leaves
\item Salt to taste
\end{enumerate}

\hrulefill

\textbf{Method}

\begin{enumerate}
\item Fry 1 tsp of mustard seeds with a lid over the top  (until they pop) in some vegetable oil. 
\item Add 10 curry leaves, 1 tsp of cumin, 1 tsp of turmeric and half a tsp of chilli 
\item Add 1 clove of garlic (crushed) and a small knob of grated ginger. 
\item Wait til pan cools, then gradually stir in some greek yoghurt (e.g. 1-2 cups)
\item Add salt to taste. Yoghurt should become a lovely yellow colour. Sauce can be added to barbecued fish, chicken, or cooked chickpeas, or can be used as a dip.
\end{enumerate}


    %\begin{figure}[h!]
    %\centering
    %\addvbuffer[12pt 8pt]{\includegraphics[width=0.65\textwidth]{../images/image.jpg}}
    %\label{fig:recipe}
    %\caption{Template for adding an image}
    %\end{figure}

\vfill
\pagebreak