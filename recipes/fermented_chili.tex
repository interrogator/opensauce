%!TEX root = ../sauce_book.tex

% recipe title
\recipe{Fermented chilli sauce}

% author in small caps
Submitted by: \textsc{Daniel McDonald, University of Melbourne}

% enticement:

\begin{shadequote*}
``'Nothing beats \#openauce'', he argued, fermentedly.
\end{shadequote*}
\hrulefill

% ingredients:
\textbf{Ingredients}

\begin{enumerate}[before=\itshape,font=\normalfont]
\item Chillis (of your favourite variety)
\item Garlic (offensive quantity)
\item Raw sugar
\item White or brown vinegar
\item Salt
\item Hip mason jars
\end{enumerate}

\hrulefill

\textbf{Method}

Fermentation is a surprisingly joyous and meditative process. Fermentation not only renders your sauce tasty and shelf-stable, but also involves the growing and killing of tiny creatures.

\begin{enumerate}
    \item Buy as many chillis as you can carry home from the market. Any variety is fine, but the sauce will be especially tasty if you use habeneros or jalape{\~n}os.
    \item Use a food processer to turn , plus garlic, into a paste. Chopping them up by hand is not recommended, as it takes a very long time, chilli juice will be lost, and you will accidentally touch your eyes at some point and be in agony.
    \item Once chillis and garlic are turned into a paste, Add salt until the sauce tastes slightly undersalted. Because you will eventually reduce the sauce, you should not salt to taste.
    \item Add raw sugar. Add so much that the sauce is about 30 per cent sweeter than you like, because the fermentation process will reduce the sweetness quite drastically.
    \item Pour paste into hip mason jars, and cover with something that breathes, like cheesecloth. Put the jar(s) far away from the reach of pets and children.
    \item After uni, muddle the sauce a little with a spoon. Don't leave the spoon in the sauce, though.
    \item After a few days, you should begin to see bubbling. If mold grows on the surface of the sauce, simply remove it with a spoon.
    \item When bubbling has subsided (usually after aroun two weeks), your sauce has fermented. Put it in a pot and bring it to a near-boil. Try to hold your breath.
    \item Add vinegar to the pot. White and brown vinegars are both alright. White may do more to preserve the vibrancy of your sauce colour, but brown may result in a more complex flavour.
    \item Allow your sauce to reduce slowly. Hold your breath here too.
    \item Once sauce is of a desirably consistency, taste it, and add salt as needed.
    \item Return sauce to hip mason jar. It can now life in your pantry.
\end{enumerate}



    %\begin{figure}[h!]
    %\centering
    %\addvbuffer[12pt 8pt]{\includegraphics[width=0.65\textwidth]{images/marinara.jpg}}
    %\label{fig:forum}
    %\caption{Stolen marinara sauce image}
    %\end{figure}

\vfill
\pagebreak


